%================================
\begin{grafico}[H]
    \centering
    % --- Subfigura A: Aceleração Positiva ---
    \begin{subgrafico}[b]{0.45\textwidth}
        \centering
        \begin{tikzpicture}
            \begin{axis}[
                width=\textwidth,
                axis lines = middle,
                xlabel = {$t$},
                ylabel = {$s$},
                xmin=0, xmax=4,
                ymin=0, ymax=6,
                xtick=\empty, ytick=\empty,
                title={Aceleração Positiva ($a > 0$)},
                title style={yshift=1ex}
            ]
                % Parábola com concavidade voltada para cima
                \addplot[thick, azul, domain=0.5:3.5, samples=100] {2*(x-2)^2 + 1};
                %\node[azul] at (axis cs: 2, 4) {\Large $\cup$};
            \end{axis}
        \end{tikzpicture}
        \caption{Concavidade voltada para cima.}
        \label{fig:concavidade_positiva}
    \end{subgrafico}
    \hfill
    % --- Subfigura B: Aceleração Negativa ---
    \begin{subgrafico}[b]{0.45\textwidth}
        \centering
        \begin{tikzpicture}
            \begin{axis}[
                width=\textwidth,
                axis lines = middle,
                xlabel = {$t$},
                ylabel = {$s$},
                xmin=0, xmax=4,
                ymin=0, ymax=6,
                xtick=\empty, ytick=\empty,
                title={Aceleração Negativa ($a < 0$)},
                title style={yshift=1ex}
            ]
                % Parábola com concavidade voltada para baixo
                \addplot[thick, vermelho, domain=0.5:3.5, samples=100] {-2*(x-2)^2 + 5};
                %\node[vermelho] at (axis cs: 2, 2) {\Large $\cap$};
            \end{axis}
        \end{tikzpicture}
        \caption{Concavidade voltada para baixo.}
        \label{fig:concavidade_negativa}
    \end{subgrafico}
    \caption{Análise comparativa da concavidade da parábola em função do sinal da aceleração escalar.}
    \label{fig:comparativo_concavidade}
\end{grafico}