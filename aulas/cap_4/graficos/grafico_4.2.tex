\begin{grafico}[H]
    \centering
    \begin{tikzpicture}
    \begin{axis}[
        axis lines = left,
        xlabel = {$t$},
        ylabel = {$v$},
        xmin=0, xmax=5,
        ymin=0, ymax=6,
        xtick={0, 4},
        xticklabels={0, $t$},
        ytick={2, 5},
        yticklabels={$v_0$, $v$},
        extra x ticks={0},
        extra x tick labels={0},
        grid style=dashed,
    ]
        % Área do deslocamento (Trapézio)
        \addplot[fill=azul!20, draw=none, domain=0:4] {0.75*x + 2} \closedcycle;
        
        % Linha da Velocidade
        \addplot[thick, azul, domain=0:4.5] {0.75*x + 2} 
            node[
                sloped,         % Faz o texto seguir a inclinação da função
                above,          % Posiciona acima da linha
                pos=0.4,        % Posição ao longo da reta (0 a 1)
                anchor=south    % Garante que a base do texto aponte para a linha
            ] {$v = v_0 + at$};
        
        % Linhas de projeção tracejadas
        \draw[dashed] (axis cs:0,5) -- (axis cs:4,5);
        \draw[dashed] (axis cs:4,0) -- (axis cs:4,5);
        
        % Divisão Retângulo/Triângulo para facilitar a dedução
        \draw[dotted, thick, vermelho] (axis cs:0,2) -- (axis cs:4,2);
        
        % Rótulos das Áreas
        \node at (axis cs:2,1) {\textbf{$A_1$}}; % Área do retângulo (v0 * t)
        \node at (axis cs:2.6,3) {\textbf{$A_2$}}; % Área do triângulo (1/2 a t^2)
        
    \end{axis}
    \end{tikzpicture}
    \caption{No grafico $v \times t$ é representado o movimento retilíneo uniformemente variado (MRUV). A área sob a curva representa o deslocamento $\Delta s$.}
    \label{fig:grafico_area_mruv}
\end{grafico}