%=======================================
\section{Questões:}

\begin{enumerate}
    \item Explique por que a criação do Sistema Internacional de Unidades (SI) foi um marco importante para o desenvolvimento tecnológico e comercial entre as nações.
    \item Diferencie fenômeno físico de fenômeno químico, citando um exemplo do cotidiano para cada um que não tenha sido mencionado no texto.
    \item No estudo da ordem de grandeza, por que utilizamos o valor de $\sqrt{10} \approx 3,16$ como critério de corte?
    \item O que define se um número está ou não escrito corretamente em notação científica? Cite o intervalo permitido.
\end{enumerate}

%=======================================
\section{Exercícios:}

\subsection*{Notação Científica e Ordem de Grandeza}
\begin{enumerate}[resume]
    \item Escreva os valores abaixo em notação científica:
        \begin{enumerate}
            \item $0,00000000025$ \unit{m} (raio atômico)
            \item $149.600.000.000$ \unit{m} (distância Terra-Sol)
            \item $0,000001$ \unit{s} (tempo de um processo eletrônico)
            \item $5.970.000.000.000.000.000.000.000$ \unit{kg} (massa da Terra)
            \item $0,00000000016$ \unit{C} (carga elementar)
        \end{enumerate}
    \item Determine a Ordem de Grandeza (OG) para cada item do exercício anterior.
\end{enumerate}

\subsection*{Conversão de Unidades (SI)}
\begin{enumerate}[resume]
    \item Converta as seguintes medidas para a unidade padrão do SI utilizando Regra de Três:
        \begin{enumerate}
            \item \qty{1500}{cm} para metros.
            \item \qty{2,5}{km} para metros.
            \item \qty{300}{min} para segundos.
            \item \qty{0,8}{kg} para gramas (conversão auxiliar).
            \item \qty{2}{h} para segundos.
        \end{enumerate}
    \item Utilize o Método do Fator de Conversão (cancelamento) para converter:
        \begin{enumerate}
            \item \qty{108}{km/h} para \unit{m/s}.
            \item \qty{5}{m/s} para \unit{km/h}.
            \item \qty{2}{g/cm^3} para \unit{kg/m^3}.
            \item \qty{1}{dia} para segundos.
            \item \qty{10}{m^2} para \unit{cm^2}.
        \end{enumerate}
\end{enumerate}

%=======================================
\section{Problemas:}

\begin{enumerate}[resume]
    \item \textbf{(Dados Reais)} A luz viaja no vácuo a uma velocidade constante de aproximadamente \qty{300000}{km/s}. 
    \begin{enumerate}
        \item Converta esta velocidade para metros por segundo (\unit{m/s}) usando notação científica.
        \item Qual a Ordem de Grandeza desta velocidade no SI?
    \end{enumerate}

    \item \textbf{(Estimativa)} Um estudante consome, em média, $2$ litros de água por dia. Quantos mililitros (\unit{ml}) ele terá consumido ao final de um ano ($365$ dias)? Expresse em notação científica.

    \item \textbf{(Desafio de Fermi)} Estime quantas vezes o coração de um adolescente bate em um único dia, considerando uma frequência média de \qty{75}{batimentos/min}. Apresente o resultado e sua Ordem de Grandeza.

    \item \textbf{(Dados Reais)} A espessura de uma folha de papel comum é de aproximadamente \qty{0,1}{mm}. 
    \begin{enumerate}
        \item Expresse essa espessura em metros (\unit{m}) usando notação científica.
        \item Se empilharmos $1$ milhão ($10^6$) dessas folhas, qual será a altura da pilha em quilômetros (\unit{km})?
    \end{enumerate}
\end{enumerate}