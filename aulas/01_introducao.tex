\chapter{Introdução à Física: Grandezas Físicas}
\label{cap:introducao}

A Física é a ciência que estuda os fenômenos fundamentais da natureza, buscando leis gerais que descrevam o comportamento da matéria e da energia \parencite{nussenzveig2002}.

Compreender o funcionamento do universo requer a medição precisa de diversas propriedades físicas, conhecidas como \textbf{grandezas físicas}. Por exemplo, a \textit{massa}, o \textit{tempo} e o \textit{comprimento} são grandezas fundamentais que descrevem aspectos essenciais dos objetos e eventos ao nosso redor. 

\section{Fenômenos Observáveis}
Os fenômenos da natureza podem ser classificados em dois tipos principais: 
\begin{itemize}
    \item \textbf{Fenômeno Físico:} Ocorre sem alterar a estrutura íntima da matéria. Exemplo: A queda de um corpo ou a mudança de estado físico da água.
    \item \textbf{Fenômeno Químico:} Altera a composição da matéria. Exemplo: A combustão de uma madeira.
\end{itemize}

\section{O Sistema Internacional de Unidades (SI)}
Para que a comunicação científica seja universal, utiliza-se o SI (Sistema Internacional de Unidades de Medida). Esse sistema foi criado com o intuito de padronizar e calibrar instrumentos no mundo todo. Essas grandezas são importantíssimas; vale a pena decorar, guardar e destacar, pois as revisitaremos no decorrer do ano. Na tabela \ref{tab:SI}, podemos observar as unidades que estudaremos:

\begin{table}[H]
    \centering
    \begin{threeparttable}
        \caption{Grandezas Fundamentais no SI}
        \label{tab:SI}
        \sffamily
        \begin{tabular}{llc}
            \toprule
            \textbf{Grandeza} & \textbf{Unidade (SI)}           & \textbf{Símbolo} \\
            \midrule
            Comprimento       & metro                          & \unit{m}         \\
            Massa             & quilograma                     & \unit{kg}        \\
            Tempo             & segundo                        & \unit{s}         \\
            Velocidade        & metro por segundo              & \unit{m/s}       \\
            Aceleração        & metro por segundo ao quadrado  & \unit{m/s^2}     \\
            Força             & newton                         & \unit{N}         \\
            Energia           & joule                          & \unit{J}         \\
            \bottomrule
        \end{tabular}
        \begin{tablenotes}[flushleft]
            \small
            \item \textbf{Nota:} Para uma lista completa e detalhada, consulte o \hyperref[apend:unidades_SI]{Apêndice~\ref{apend:unidades_SI}}.
        \end{tablenotes}
    \end{threeparttable}
\end{table}

\subsection{Conversão de Unidades e Regra de Três}

Nem sempre os dados de um problema estarão no SI. Para converter unidades de forma segura, a **regra de três simples** é a ferramenta mais confiável, pois utiliza uma relação de equivalência conhecida (fator de conversão) para encontrar o valor desejado.



**Exemplo: Converter 250 cm para metros.**
Sabemos que $1$ \unit{m} equivale a $100$ \unit{cm}. Montamos a proporção alinhando as unidades iguais:

\begin{align*}
    1 \text{ m}   & \text{ --- } 100 \text{ cm} \\
    x \text{ m}   & \text{ --- } 250 \text{ cm} \\
    100 \cdot x   & = 1 \cdot 250 \implies x = \frac{250}{100} = 2,50 \text{ m}
\end{align*}

Para conversões comuns, consulte a tabela de fatores no \hyperref[apend:fatores_conversao]{Apêndice~\ref{apend:fatores_conversao}}.

\subsection{Método do Fator de Conversão (Cancelamento de Unidades)}

Para conversões múltiplas ou unidades compostas, o método mais eficiente é multiplicar o valor original por uma fração de equivalência (fator de conversão). O objetivo é posicionar a unidade que queremos "cortar" no lado oposto da fração.

**Exemplo: Converter a velocidade de \qty{72}{km/h} para \unit{m/s}.**

Precisamos de dois fatores: um para distância (\unit{1}{km} = \unit{1000}{m}) e um para tempo (\unit{1}{h} = \unit{3600}{s}).

\begin{equation*}
    v = 72 \text{ \cancel{km}} / \text{\cancel{h}} \cdot \underbrace{\left( \frac{1000 \text{ m}}{1 \text{ \cancel{km}}} \right)}_{\text{Fator 1}} \cdot \underbrace{\left( \frac{1 \text{ \cancel{h}}}{3600 \text{ s}} \right)}_{\text{Fator 2}}
\end{equation*}

Calculando os valores:
\begin{equation*}
    v = \frac{72 \cdot 1000 \text{ m}}{3600 \text{ s}} = \frac{72000}{3600} \text{ m/s} = 20,0 \text{ m/s}
\end{equation*}

Este método reduz drasticamente a chance de erro ao lidar com unidades de área ($\unit{m^2}$) ou volume ($\unit{m^3}$), pois basta elevar o fator de conversão à potência desejada. Observe:
\begin{equation*}
    1 \text{ m}^2 = \left( \frac{100 \text{ cm}}{1 \text{ m}} \right)^2 = \frac{100^2 \text{ cm}^2}{1^2 \text{ m}^2} = 10.000 \text{ cm}^2  
\end{equation*}

\section{Notação Científica}
Na física, lidamos com dimensões que variam do átomo às galáxias. Utilizamos a notação científica na forma geral \ref{eq:notacao_cientifica} para expressar esses valores de maneira compacta e clara.
\begin{equation}
    N \times 10^{n}, \quad \text{sendo } 1 \leq N < 10 \label{eq:notacao_cientifica}
\end{equation}

\textbf{Aplicações:} Expressar a velocidade da luz ($3 \times 10^8$ m/s) ou a massa de um elétron de forma simplificada.

Na prática, o objetivo é reduzir o número de zeros, facilitando cálculos e comparações. Tudo que precisa ser feito é deslocar a vírgula para a direita ou esquerda, dependendo se o número é maior ou menor que 10. Como podemos ver:
\begin{itemize}
    \item Para números menores que 10, deslocamos a vírgula para a direita, o que resulta em um expoente negativo. Cada casa deslocada corresponde a uma unidade a mais no expoente negativo.
        \begin{enumerate}
            \item Deslocar a vírgula em 1 casa: $0,52 = 5,2 \times 10^{-1}$
            \item Deslocar a vírgula em 2 casas: $0,0502 = 5,02 \times 10^{-2}$
            \item Deslocar a vírgula em 3 casas: $0,008 = 8 \times 10^{-3}$
        \end{enumerate}
        \textbf{Observe que o número antes do "x10"  sempre fica entre 1 e 10.}
    \item Para números maiores que 10, deslocamos a vírgula para a esquerda, o que resulta em um expoente positivo. Cada casa deslocada corresponde a uma unidade a mais no expoente positivo. 
        \begin{enumerate}
            \item Deslocar a vírgula em 1 casa: $52 = 5,2 \times 10^{1}$
            \item Deslocar a vírgula em 2 casas: $502 = 5,02 \times 10^{2}$
            \item Deslocar a vírgula em 3 casas: $4500000 = 4,5 \times 10^{6}$
        \end{enumerate}
        \textbf{Observe que o zero entre dois números, deve ser expresso após a vírgula e entra na contagem.}
\end{itemize}

\section{Ordem de Grandeza}
A ordem de grandeza é uma forma de classificar a magnitude de um número, expressando-a  numa potência de 10. Ela é útil para comparar quantidades e entender a escala dos fenômenos físicos. A ordem de grandeza é determinada pelo expoente na notação científica mais o fator de proximidade logarítmico.
\begin{itemize}
    \item Se o número $N$ na notação científica $N \times 10^{n}$ for menor que $\sqrt{10}$, a ordem de grandeza é o expoente $n$ como podemos ver em \ref{eq:og_pequeno}.
       
        \begin{equation}
            N < \sqrt{10} \implies \text{[OG]} = n \label{eq:og_pequeno}
        \end{equation}

        Lembrando que n é o expoente da base 10.

        \textbf{Exemplo 1:} $3,2 \times 10^{4}$, como $3,1 < \sqrt{10}$ (3,1 menor que ..), a ordem de grandeza é 4.

        Resposta: [OG]=4. (OG = Ordem de Grandeza)

        \textbf{Exemplo 2:} $1,7 \times 10^{-3}$, como $1,7 < \sqrt{10}$ (1,7 menor que ..), a ordem de grandeza é -3.
    
        Resposta: [OG]=-3.

    \item Se o número $N$ for maior ou igual a $\sqrt{10}$, a ordem de grandeza é o expoente $n + 1$ como podemos ver \ref{eq:og_grande}.
        
        \begin{equation}
            N > \sqrt{10} \implies \text{[OG]} = n + 1 \label{eq:og_grande}
        \end{equation}
        
        \textbf{Exemplo 1:} $6,5 \times 10^{8}$, como $6,5 \geq \sqrt{10}$ (6,5 maior que ..), a ordem de grandeza é $8 + 1 = 9$.

        Resposta: [OG]=9.
    
        \textbf{Exemplo 2:} $8,9 \times 10^{-4}$, como $8,9 \geq \sqrt{10}$ (8,9 maior que ..), a ordem de grandeza é $-4 + 1 = -3$.
   
        Resposta: [OG]=-3.
\end{itemize}
\textbf{Observação:} A raiz quadrada de 10 ($\sqrt{10}$) é aproximadamente 3,16. Portanto, ao determinar a ordem de grandeza, compare o valor de $N$ com 3,16 para decidir se deve usar o expoente ou somar 1. Segundo detalhe, cuidado com números negativos, pois ao somar 1 em um número negativo, o valor se aproxima de zero.

\section{Questões Conceituais}

\begin{enumerate}
    \item Explique por que a criação do Sistema Internacional de Unidades (SI) foi um marco importante para o desenvolvimento tecnológico e comercial entre as nações.
    \item Diferencie fenômeno físico de fenômeno químico, citando um exemplo do cotidiano para cada um que não tenha sido mencionado no texto.
    \item No estudo da ordem de grandeza, por que utilizamos o valor de $\sqrt{10} \approx 3,16$ como critério de corte?
    \item O que define se um número está ou não escrito corretamente em notação científica? Cite o intervalo permitido.
\end{enumerate}

\clearpage

\section{Exercícios de Fixação}

\subsection*{Notação Científica e Ordem de Grandeza}
\begin{enumerate}[resume]
    \item Escreva os valores abaixo em notação científica:
        \begin{enumerate}
            \item $0,00000000025$ \unit{m} (raio atômico)
            \item $149.600.000.000$ \unit{m} (distância Terra-Sol)
            \item $0,000001$ \unit{s} (tempo de um processo eletrônico)
            \item $5.970.000.000.000.000.000.000.000$ \unit{kg} (massa da Terra)
            \item $0,00000000016$ \unit{C} (carga elementar)
        \end{enumerate}
    \item Determine a Ordem de Grandeza (OG) para cada item do exercício anterior.
\end{enumerate}

\subsection*{Conversão de Unidades (SI)}
\begin{enumerate}[resume]
    \item Converta as seguintes medidas para a unidade padrão do SI utilizando Regra de Três:
        \begin{enumerate}
            \item \qty{1500}{cm} para metros.
            \item \qty{2,5}{km} para metros.
            \item \qty{300}{min} para segundos.
            \item \qty{0,8}{kg} para gramas (conversão auxiliar).
            \item \qty{2}{h} para segundos.
        \end{enumerate}
    \item Utilize o Método do Fator de Conversão (cancelamento) para converter:
        \begin{enumerate}
            \item \qty{108}{km/h} para \unit{m/s}.
            \item \qty{5}{m/s} para \unit{km/h}.
            \item \qty{2}{g/cm^3} para \unit{kg/m^3}.
            \item \qty{1}{dia} para segundos.
            \item \qty{10}{m^2} para \unit{cm^2}.
        \end{enumerate}
\end{enumerate}

\clearpage

\section{Problemas Aplicados}

\begin{enumerate}[resume]
    \item \textbf{(Dados Reais)} A luz viaja no vácuo a uma velocidade constante de aproximadamente \qty{300000}{km/s}. 
    \begin{enumerate}
        \item Converta esta velocidade para metros por segundo (\unit{m/s}) usando notação científica.
        \item Qual a Ordem de Grandeza desta velocidade no SI?
    \end{enumerate}

    \item \textbf{(Estimativa)} Um estudante consome, em média, $2$ litros de água por dia. Quantos mililitros (\unit{ml}) ele terá consumido ao final de um ano ($365$ dias)? Expresse em notação científica.

    \item \textbf{(Desafio de Fermi)} Estime quantas vezes o coração de um adolescente bate em um único dia, considerando uma frequência média de \qty{75}{batimentos/min}. Apresente o resultado e sua Ordem de Grandeza.

    \item \textbf{(Dados Reais)} A espessura de uma folha de papel comum é de aproximadamente \qty{0,1}{mm}. 
    \begin{enumerate}
        \item Expresse essa espessura em metros (\unit{m}) usando notação científica.
        \item Se empilharmos $1$ milhão ($10^6$) dessas folhas, qual será a altura da pilha em quilômetros (\unit{km})?
    \end{enumerate}
\end{enumerate}

\clearpage