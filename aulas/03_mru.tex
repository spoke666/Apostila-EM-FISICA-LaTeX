\chapter{Movimento Retilíneo Uniforme (MRU)}
\label{cap:mru}

O Movimento Retilíneo Uniforme (MRU) é o caso mais simples de movimento, servindo como base para a compreensão de dinâmicas mais complexas. Sua principal característica é a manutenção de uma velocidade constante ao longo de uma trajetória reta.

\section{Definição e Características}

No MRU, o móvel percorre distâncias iguais em intervalos de tempo iguais. Isso implica que:
\begin{itemize}
    \item A trajetória é uma linha reta.
    \item A velocidade escalar instantânea é constante e igual à velocidade escalar média ($v = v_m$).
    \item A aceleração é nula ($a = 0$), pois não há variação no módulo da velocidade.
\end{itemize}

\section{Função Horária da Posição}

A função horária permite determinar a posição ($s$) de um móvel em qualquer instante ($t$), desde que conheçamos sua posição inicial ($s_0$) e sua velocidade ($v$). Partindo da definição de velocidade média:

\begin{equation}
    v = \frac{\Delta s}{\Delta t} \implies v = \frac{s - s_0}{t - t_0}
\end{equation}

Considerando o instante inicial como $t_0 = 0$, temos:
\begin{equation}
    v = \frac{s - s_0}{t} \implies v \cdot t = s - s_0
\end{equation}

Isolando a posição final ($s$), obtemos a \textbf{Função Horária do MRU}\parencite{halliday2012}:

\begin{equation}
    s = s_0 + v \cdot t \label{eq:sorvete}
\end{equation}

\noindent Onde:
\begin{itemize}
    \item $s$: Posição final no instante $t$ (\unit{m}).
    \item $s_0$: Posição inicial no instante $t=0$ (\unit{m}).
    \item $v$: Velocidade escalar constante (\unit{m/s}).
    \item $t$: Tempo decorrido (\unit{s}).
\end{itemize}

\begin{example}
Um móvel parte da posição \unit{20}{m} com uma velocidade constante de \unit{5}{m/s}. Sua função horária será $s = 20 + 5t$. No instante $t = \unit{10}{s}$, sua posição será:
\begin{equation*}
    s = 20 + 5(10) = 20 + 50 = \unit{70}{m}
\end{equation*}
\end{example}

\section{Análise de Dados e Tabelas}

A identificação de um Movimento Retilíneo Uniforme na prática ocorre através da coleta de posições em intervalos de tempo conhecidos. A principal característica matemática para identificar o MRU em uma tabela é observar que, para intervalos de tempo iguais ($\Delta t$), o deslocamento ($\Delta s$) também deve ser igual.

\subsection{Identificação Experimental}

Considere um móvel cujas posições foram registradas conforme a tabela abaixo:

\begin{table}[H]
    \centering
    \begin{tabular}{ccccccc}
        \toprule
        \textbf{Tempo} $t$ (s) & 0 & 1 & 2 & 3 & 4 & 5 \\ \midrule
        \textbf{Posição} $s$ (m) & 10 & 15 & 20 & 25 & 30 & 35 \\ \bottomrule
    \end{tabular}
    \caption{Registro de posições de um móvel em MRU.}
    \label{tab:dados_mru}
\end{table}

Ao analisarmos os dados da Tabela \ref{tab:dados_mru}, notamos que:
\begin{itemize}
    \item No instante $t=0$, a posição inicial é $s_0 = \unit{10}{m}$.
    \item A cada \unit{1}{s} que passa, a posição aumenta exatamente \unit{5}{m}.
    \item A razão $\frac{\Delta s}{\Delta t} = \frac{5}{1} = \unit{5}{m/s}$ é constante.
\end{itemize}

\subsection{Construção da Função Horária a partir de Dados}

Para montar a função horária $s = s_0 + v \cdot t$ a partir de uma tabela, seguimos dois passos simples:
\begin{enumerate}
    \item Identificamos $s_0$ observando o valor de $s$ quando $t=0$.
    \item Calculamos a velocidade $v$ escolhendo quaisquer dois pontos da tabela e aplicando $v = \frac{s_2 - s_1}{t_2 - t_1}$.
\end{enumerate}

No caso da tabela anterior:
\begin{equation*}
    s_0 = \unit{10}{m} \quad \text{e} \quad v = \frac{20 - 15}{2 - 1} = \unit{5}{m/s}
\end{equation*}
Logo, a função horária que descreve esses dados é: $s = 10 + 5t$.