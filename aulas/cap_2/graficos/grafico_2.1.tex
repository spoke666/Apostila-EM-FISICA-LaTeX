% --- Exemplo de Sistema Unidimensional ---
\begin{figure}[htbp]
    \centering
    % --- Subfigura A: Unidimensional ---
    \begin{subfigure}[b]{0.45\textwidth}
        \centering
        \begin{tikzpicture}[>=stealth, scale=0.8]
            \draw[->, thick] (-2.5,0) -- (2.5,0) node[right] {$s$};
            \foreach \x in {-2,-1,0,1,2}
                \draw (\x, 0.1) -- (\x, -0.1) node[below] {\scriptsize \x};
            \filldraw[azul] (0,0) circle (2pt) node[above=2pt] {\scriptsize $O$};
            \filldraw[vermelho] (1.5,0) circle (2pt);
        \end{tikzpicture}
        \caption{Sistema Unidimensional (Reta)}
        \label{fig:coord_unidim}
    \end{subfigure}
    \hfill % Espaçamento entre as figuras
    % --- Subfigura B: Bidimensional ---
    \begin{subfigure}[b]{0.45\textwidth}
        \centering
        \begin{tikzpicture}[>=stealth, scale=0.6]
            \draw[->, thick] (-1,0) -- (3,0) node[right] {$x$};
            \draw[->, thick] (0,-1) -- (0,3) node[above] {$y$};
            \draw[help lines, dashed] (0,0) grid (2,2);
            \filldraw[azul] (0,0) circle (3pt) node[below left] {\scriptsize $O$};
            \filldraw[vermelho] (2,2) circle (3pt) node[above right] {\scriptsize $P(2,2)$};
        \end{tikzpicture}
        \caption{Sistema Bidimensional (Plano)}
        \label{fig:coord_bidim}
    \end{subfigure}

    \vspace{0.3cm}
    \caption{Representação visual dos sistemas de coordenadas e suas respectivas origens ($O$).}
    \label{fig:sistemas_coordenadas}
\end{figure}

\begin{itemize}
    \item \textbf{Movimento:} Variação da posição ao longo do tempo em relação ao referencial.
    \item \textbf{Repouso:} Posição sem alteração ao longo do tempo de acordo com o referencial adotado.
\end{itemize}

% Exemplo de relatividade do movimento
\begin{figure}[htbp]
    \centering
    \includegraphics[width=0.8\textwidth]{fig-2.1.png}
    \caption[Movimento relativo no ônibus]{Ilustração do movimento relativo de uma esfera: Na figura superior, nota-se um passageiro observando o movimento vertical de uma esfera. Na figura inferior, nota-se um observador externo visualizando uma trajetória parabólica da mesma esfera.}
    \label{fig:movimento_relativo}
\end{figure}