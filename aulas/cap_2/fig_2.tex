\newcommand{\figBa}{
% Exemplo de relatividade do movimento
    \begin{figure}[htbp]
        \centering
        \includegraphics[width=0.8\textwidth]{fig-2.1.png}
        \caption[Movimento relativo no ônibus]{Ilustração do movimento relativo de uma esfera: Na figura superior, nota-se um passageiro observando o movimento vertical de uma esfera. Na figura inferior, nota-se um observador externo visualizando uma trajetória parabólica da mesma esfera.}
        \label{fig:movimento_relativo}
    \end{figure}
    
}

\newcommand{\figBb}{
    % Exemplo de ponto material vs corpo extenso
    \begin{figure}[htbp]
        \centering
        \includegraphics[width=0.6\textwidth]{fig-2.2.png}
        \caption[Movimento relativo no ônibus]{A figura demonstra a diferença entre um ponto material e um corpo extenso. À esquerda, uma pessoa sob uma superfície muito extensa. A direita, uma pessoa em uma superfície pequena, onde suas dimensões são relevantes.}
        \label{fig:corpo_ponto}
    \end{figure}    
}

\newcommand{\figBc}{
    \begin{figure}[H]
        \centering
        % Primeira subfigura: ocupa 45% da largura da linha
        \begin{subfigure}{0.45\textwidth}
            \centering
            \includegraphics[width=\textwidth]{fig-2.3(a).png}
            \caption{Movimento Progressivo.}
            \label{fig:prog}
        \end{subfigure}
        \hfill % Adiciona um espaço elástico entre as duas
        % Segunda subfigura: ocupa 45% da largura da linha
        \begin{subfigure}{0.45\textwidth}
            \centering
            \includegraphics[width=\textwidth]{fig-2.3(b).png}
            \caption{Movimento Retrógrado.}
            \label{fig:ret}
        \end{subfigure}
        \caption{Sentidos de movimento em relação à trajetória.}
    \end{figure}
}

\newcommand{\figBd}{
    % Exemplo de velocidade instantânea
    \begin{figure}[htbp]
        \centering
        \includegraphics[width=0.4\textwidth]{fig-2.4.png}
        \caption[Velocímetro]{No velocímetro, a velocidade instantânea é indicada pela posição da agulha em um dado instante, refletindo a rapidez do veículo naquele momento específico.}
        \label{fig:velocimetro}
    \end{figure}
}

\newcommand{\figBe}{
    % Exemplo de velocidade média
    \begin{figure}[htbp]
        \centering
        \includegraphics[width=0.4\textwidth]{fig-2.5.png}
        \caption[Veículo em trajeto curvilíneo]{Na figura é retratado algumas manobras que um veículo pode realizar. Observe que, o veículo sai da posição $S_0$ no tempo $t_0$ e percorre um trajeto curvilíneo até chegar na posição $S$ no tempo $t$.}
        \label{fig:veiculo_trajeto}
    \end{figure}
}