%=======================================
\section{Questões:}

\begin{enumerate}
    \item No estudo da cinemática, dizemos que o repouso e o movimento são conceitos relativos. Explique o que isso significa e qual a importância do \textbf{referencial} nessa definição.
    
    \item Imagine um passageiro dentro de um trem que se desloca com velocidade constante em uma linha férrea retilínea. O passageiro solta uma moeda. 
    \begin{enumerate}[label=\alph*)]
        \item Qual a trajetória da moeda para o passageiro?
        \item Qual a trajetória da moeda para um observador parado na plataforma da estação?
    \end{enumerate}

    \item Diferencie \textbf{ponto material} de \textbf{corpo extenso}. É possível que um mesmo objeto (como um transatlântico) seja considerado ponto material em uma situação e corpo extenso em outra? Justifique.

    \item Um motorista observa o velocímetro de seu carro e nota que a agulha marca exatamente \unit{100}{km/h}. Esse valor refere-se à velocidade escalar média ou à velocidade escalar instantânea? Justifique sua resposta.

    \item O que caracteriza um movimento como \textbf{retrógrado}? Nesse caso, o valor da velocidade escalar média será positivo ou negativo?

    \item Um atleta completa uma volta em uma pista circular de \unit{400}{m}. 
    \begin{enumerate}[label=\alph*)]
        \item Qual foi o deslocamento ($\Delta S$) do atleta ao final da volta?
        \item A distância percorrida por ele é igual ao seu deslocamento? Explique.
    \end{enumerate}

    \item No Sistema Internacional de Unidades (SI), qual é a unidade padrão para a velocidade? Por que no cotidiano é mais comum utilizarmos o \unit{}{km/h} em vez da unidade do SI?

    \item Considere uma rodovia onde a orientação da trajetória cresce de Sul para Norte. Um carro que viaja de uma cidade ao Norte para uma cidade ao Sul está realizando um movimento progressivo ou retrógrado? Por quê?
\end{enumerate}

%=======================================
\section{Exercícios}

\begin{enumerate}
    \item Um automóvel percorre uma distância de \unit{450}{km} em \unit{5}{h}. Calcule a velocidade escalar média do veículo nesse trajeto em \unit{}{km/h}.
    
    \item Um móvel parte da posição $S_0 = \unit{15}{m}$ e, após \unit{10}{s}, encontra-se na posição $S = \unit{85}{m}$. Determine a sua velocidade média no Sistema Internacional (SI).
    
    \item Uma aeronave comercial voa com velocidade média de \unit{800}{km/h}. Quanto tempo ela levará para completar um percurso de \unit{2400}{km} entre dois aeroportos?
    
    \item Um corredor mantém uma velocidade constante de \unit{4}{m/s}. Qual a posição $S$ do atleta após \unit{1}{min} de corrida, sabendo que ele partiu da origem das posições ($S_0 = 0$)?
    
    \item Um trem de \unit{200}{m} de comprimento atravessa um túnel de \unit{300}{m} com velocidade constante de \unit{20}{m/s}. Quanto tempo o trem leva para atravessar completamente o túnel?
    
    \item Converta as seguintes velocidades de \unit{}{km/h} para \unit{}{m/s} ou vice-versa:
    \begin{enumerate}[label=\alph*)]
        \item \unit{72}{km/h}
        \item \unit{10}{m/s}
        \item \unit{108}{km/h}
        \item \unit{30}{m/s}
    \end{enumerate}

    \item Um motorista viaja de uma cidade A até uma cidade B. Ele percorre os primeiros \unit{120}{km} com velocidade de \unit{60}{km/h} e os \unit{120}{km} seguintes com velocidade de \unit{40}{km/h}. Qual a velocidade média para o trajeto total de \unit{240}{km}?
    
    \item Um protótipo de testes realiza um circuito dividido em três etapas, conforme a tabela abaixo:

    \begin{table}[H]
        \centering
        \begin{tabular}{ccc}
            \toprule
            Etapa & Velocidade ($v_i$) & Tempo ($\Delta t_i$) \\ \midrule
            A & \unit{10}{m/s} & \unit{20}{s} \\
            B & \unit{25}{m/s} & \unit{10}{s} \\
            C & \unit{15}{m/s} & \unit{10}{s} \\ \bottomrule
        \end{tabular}
    \end{table}
    
    Com base nos dados, determine a velocidade média total do protótipo ao longo de todo o circuito.

    \item Dois automóveis, $A$ e $B$, movem-se em uma estrada retilínea com velocidades constantes $v_A = \unit{80}{km/h}$ e $v_B = \unit{60}{km/h}$. Determine a velocidade relativa de $A$ em relação a $B$ nas seguintes situações:
    \begin{enumerate}[label=\alph*)]
        \item Quando os dois movem-se no mesmo sentido.
        \item Quando os dois movem-se em sentidos opostos.
    \end{enumerate}

    \item Dois trens de carga deslocam-se em trilhos paralelos no mesmo sentido. O trem $A$ tem comprimento de \unit{150}{m} e velocidade de \unit{15}{m/s}, enquanto o trem $B$ tem \unit{100}{m} e velocidade de \unit{10}{m/s}. Quanto tempo o trem $A$ leva para ultrapassar completamente o trem $B$?

    \item Dois ciclistas partem simultaneamente de dois pontos de uma ciclovia, distantes \unit{500}{m} um do outro. O ciclista $A$ parte da posição $S_{0A} = \unit{0}{m}$ com velocidade constante de \unit{8}{m/s}, e o ciclista $B$ parte de $S_{0B} = \unit{500}{m}$ com velocidade constante de \unit{12}{m/s}, em sentidos opostos (indo um de encontro ao outro).
    \begin{enumerate}[label=\alph*)]
        \item Qual o módulo da velocidade relativa de aproximação entre os ciclistas?
        \item Após quanto tempo ocorrerá o encontro?
        \item Em que posição $S$ da ciclovia eles se encontrarão?
    \end{enumerate}

    \item Um barco tenta atravessar um rio perpendicularmente à margem com uma velocidade própria de \unit{4}{m/s}. Sabendo que a velocidade da correnteza do rio é de \unit{3}{m/s}, determine o módulo da velocidade resultante do barco em relação à margem (referencial fixo na terra).

\end{enumerate}

%=======================================
\section{Problemas}

\begin{enumerate}
    \item \textbf{(Encontro de Móveis)} Dois ciclistas, $A$ e $B$, partem simultaneamente das posições $S_{0A} = \unit{0}{m}$ e $S_{0B} = \unit{2000}{m}$ de uma ciclovia retilínea. O ciclista $A$ move-se com velocidade constante $v_A = \unit{15}{m/s}$ e o ciclista $B$ com $v_B = \unit{10}{m/s}$, ambos em sentidos opostos (um de encontro ao outro).
    \begin{enumerate}[label=\alph*)]
        \item Determine o módulo da velocidade relativa de aproximação entre os ciclistas.
        \item Calcule o instante $t$ em que eles se encontram.
        \item Determine a posição $S$ do encontro em relação à origem das posições.
    \end{enumerate}

    \item \textbf{(Ultrapassagem de Corpo Extenso)} Um trem de carga possui comprimento total de \unit{250}{m} e viaja com velocidade constante de \unit{54}{km/h}. Ele deve atravessar completamente uma ponte ferroviária de \unit{150}{m} de extensão.
    \begin{enumerate}[label=\alph*)]
        \item Converta a velocidade do trem para o Sistema Internacional (\unit{}{m/s}).
        \item Qual a distância total que o trem deve percorrer para atravessar completamente a ponte?
        \item Quanto tempo, em segundos, o trem leva para concluir a travessia?
    \end{enumerate}

    \item \textbf{(Velocidade Média em Trechos)} Um motorista deseja realizar uma viagem de \unit{300}{km} com uma velocidade média total de \unit{100}{km/h}. Nos primeiros \unit{150}{km}, devido ao tráfego intenso, ele conseguiu manter uma velocidade média de apenas \unit{75}{km/h}.
    \begin{enumerate}[label=\alph*)]
        \item Qual foi o tempo gasto para percorrer a primeira metade da viagem?
        \item Qual deve ser a velocidade média na segunda metade do percurso para que o motorista consiga atingir seu objetivo inicial de \unit{100}{km/h} para a viagem toda?
    \end{enumerate}
\end{enumerate}