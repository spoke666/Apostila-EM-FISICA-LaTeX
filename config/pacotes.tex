% =============================================================================
% ARQUIVO: config/pacotes.tex (Organizado e Refatorado)
% =============================================================================

% --- 1. CORE (Sistema, Idioma e Codificação) ---
\usepackage[utf8]{inputenc}
\usepackage[T1]{fontenc}
\usepackage[brazilian]{babel}
\usepackage{csquotes} 

% --- 2. LAYOUT E ESTILO DE PÁGINA ---
\usepackage[scaled]{helvet}
\renewcommand{\familydefault}{\sfdefault}
\usepackage[top=2cm, bottom=2cm, left=2cm, right=2cm]{geometry}
\setlength{\headheight}{15pt}
\usepackage{indentfirst}
\usepackage{setspace}
\onehalfspacing 
\usepackage{fancyhdr}

% --- 3. TÍTULOS E ESTRUTURA (Refatoração de Capítulos) ---
\usepackage{titlesec}
\usepackage{titletoc}
\usepackage{etoolbox}
\usepackage{enumitem}

% --- 4. CIÊNCIAS EXATAS (Matemática, Física e Química) ---
\usepackage{amsmath, amsfonts, amssymb} 
\usepackage{bm}        % Negrito em símbolos matemáticos (Vetores)
\usepackage{siunitx}   % ESSENCIAL: Unidades do SI e alinhamento decimal
\usepackage[version=4]{mhchem} % ESSENCIAL: Fórmulas químicas (ex: \ce{H2O})
\usepackage{cancel}

% Configuração do SIUNITX para padrão brasileiro (vírgula decimal)
\sisetup{
    output-decimal-marker = {,},
    separate-uncertainty = true,
    range-phrase = { a },
    exponent-product = \cdot
}

% --- 5. TABELAS E ELEMENTOS FLUTUANTES ---
\usepackage{array}   
\usepackage{booktabs}  % Linhas padrão Halliday/Livro científico
\usepackage{float}     
\usepackage{caption}
\usepackage{threeparttable}

% --- 6. GRÁFICOS E FIGURAS ---
\usepackage{graphicx}
\usepackage{xcolor}
\usepackage{tikz}
\usepackage{pgfplots}
\pgfplotsset{compat=1.18}
\usetikzlibrary{shapes.geometric, arrows, positioning}
\usepackage{subcaption}

% --- 7. CITAÇÕES E BIBLIOGRAFIA ---
\usepackage[style=abnt, backend=biber]{biblatex}
\addbibresource{referencias.bib} 

% --- 8. LINKS E CÓDIGOS ---
\usepackage{hyperref}
\hypersetup{
    colorlinks=true, 
    allcolors=blue,
    breaklinks=true
}
\usepackage{listings}

% --- 9. DEFINIÇÃO DE CORES E LISTINGS (CÓDIGO) ---
\definecolor{azul}{RGB}{0, 0, 255}
\definecolor{vermelho}{RGB}{200, 50, 55}
\definecolor{codegreen}{rgb}{0,0.6,0}
\definecolor{codegray}{rgb}{0.5,0.5,0.5}
\definecolor{codepurple}{rgb}{0.58,0,0.82}
\definecolor{backcolour}{rgb}{0.95,0.95,0.92}

\lstset{
    backgroundcolor=\color{backcolour},   
    basicstyle=\small\ttfamily,
    keywordstyle=\color{magenta}\bfseries,
    commentstyle=\color{codegreen},
    stringstyle=\color{codepurple},
    numberstyle=\tiny\color{codegray},
    breaklines=true,
    frame=single,
    numbers=left,
    showstringspaces=false, 
    inputencoding=utf8,
    extendedchars=true,
    literate={á}{{\'a}}1 {à}{{\`a}}1 {À}{{\`A}}1 {ã}{{\~a}}1 {é}{{\'e}}1 {â}{{\^a}}1 {ê}{{\^e}}1 {í}{{\'i}}1 {ó}{{\'o}}1 {õ}{{\~o}}1 {ú}{{\'u}}1 {ç}{{\c{c}}}1 {Á}{{\'A}}1 {Ã}{{\~A}}1 {É}{{\'E}}1 {Ê}{{\^E}}1 {Í}{{\'I}}1 {Ó}{{\'O}}1 {Õ}{{\~O}}1 {Ú}{{\'U}}1 {Ç}{{\c{C}}}1 {│}{{|}}1 {├}{{+}}1 {─}{{-}}1 {└}{{+}}1
}
