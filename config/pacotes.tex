% =============================================================================
% ARQUIVO: config/pacotes.tex (Refatorado e Corrigido)
% =============================================================================

% --- 1. CORE (Sistema, Idioma e Codificação) ---
\usepackage[utf8]{inputenc}
\usepackage[T1]{fontenc}
\usepackage[brazilian]{babel}
\usepackage{csquotes} 

% --- 2. LAYOUT E ESTILO DE PÁGINA ---
\usepackage[scaled]{helvet}
\renewcommand{\familydefault}{\sfdefault}
\usepackage[top=2cm, bottom=2cm, left=2cm, right=2cm]{geometry}
\setlength{\headheight}{15pt} 
\usepackage{indentfirst}
\usepackage{setspace}
\onehalfspacing 
\usepackage{fancyhdr}
\usepackage{chngcntr} 

% --- 3. TÍTULOS E ESTRUTURA ---
\usepackage{titlesec}
\usepackage{titletoc}
\usepackage{etoolbox}
\usepackage{enumitem}
\newenvironment{example}{\begin{quote}\textbf{Exemplo:} }{\end{quote}} 

% --- 4. CIÊNCIAS EXATAS (Matemática, Física e Química) ---
\usepackage{amsmath, amsfonts, amssymb} 
\usepackage{mathtools} 
\usepackage{bm}        
\usepackage{siunitx}   
\usepackage[version=4]{mhchem} 
\usepackage{cancel}    
\usepackage{witharrows} 

% CONFIGURAÇÃO DO SIUNITX (Correção para MiKTeX 25.12 / siunitx v3)
\sisetup{
    output-decimal-marker = {,},
    separate-uncertainty = true,
    range-phrase = { a },
    exponent-product = \cdot,
    inter-unit-product = \cdot, % Ponto entre unidades (ex: m·s⁻¹)
    % Habilita os comandos \kilo, \metre, \second, \hour, etc.
    % Se o erro persistir, use as siglas diretamente: \unit{80}{km/h}
}

% --- 5. TABELAS E ELEMENTOS FLUTUANTES ---
\usepackage{array}
\usepackage{booktabs}
\usepackage{float}
\usepackage{caption}
\usepackage{subcaption} 
\usepackage{threeparttable}

% --- 6. GRÁFICOS E TIKZ ---
\usepackage{graphicx}
\usepackage{xcolor}
\usepackage{tikz}
\usepackage{pgfplots}
\pgfplotsset{compat=1.18}
\usetikzlibrary{shapes.geometric, arrows, positioning, tikzmark}

%-------Figuras-------
\graphicspath{
    {aulas/cap_1/figuras/}
    {aulas/cap_2/figuras/}
    {aulas/cap_3/figuras/}
    {aulas/cap_4/figuras/}
}

%-------Gráficos------
% No pacotes.tex
\newcommand{\caminhograficos}{aulas/cap_1/graficos} 
\newcommand{\incluirgrafico}[1]{\input{\caminhograficos/#1}}



% --- 7. CITAÇÕES E BIBLIOGRAFIA (BibLaTeX - Estilo Numérico Sobrescrito) ---
\usepackage[
    style=numeric-comp, 
    sorting=none,       
    backend=biber
]{biblatex}
\addbibresource{referencias.bib} 

% Comando para citação sobrescrita (estilo Nature/IEEE) conforme sua instrução
\DeclareCiteCommand{\parencite}[\mkbibsuperscript]
  {\usebibmacro{cite:init}%
   \usebibmacro{prenote}}
  {\usebibmacro{citeindex}%
   \usebibmacro{cite:comp}}
  {}
  {\usebibmacro{cite:dump}%
   \usebibmacro{postnote}}

% --- 8. PROGRAMAÇÃO (Listings) ---
\usepackage{listings}
\definecolor{azul}{RGB}{0, 0, 255}
\definecolor{vermelho}{RGB}{200, 50, 55}
\definecolor{codegreen}{rgb}{0,0.6,0}
\definecolor{codegray}{rgb}{0.5,0.5,0.5}
\definecolor{codepurple}{rgb}{0.58,0,0.82}
\definecolor{backcolour}{rgb}{0.95,0.95,0.92}

\lstset{
    backgroundcolor=\color{backcolour},   
    basicstyle=\small\ttfamily,
    keywordstyle=\color{magenta}\bfseries,
    commentstyle=\color{codegreen},
    stringstyle=\color{codepurple},
    numberstyle=\tiny\color{codegray},
    breaklines=true,
    frame=single,
    numbers=left,
    showstringspaces=false, 
    inputencoding=utf8,
    extendedchars=true,
    literate={á}{{\'a}}1 {à}{{\`a}}1 {ã}{{\~a}}1 {é}{{\'e}}1 {ê}{{\^e}}1 {í}{{\'i}}1 {ó}{{\'o}}1 {õ}{{\~o}}1 {ú}{{\'u}}1 {ç}{{\c{c}}}1 {Á}{{\'A}}1 {Ã}{{\~A}}1 {É}{{\'E}}1 {Ê}{{\^E}}1 {Í}{{\'I}}1 {Ó}{{\'O}}1 {Õ}{{\~O}}1 {Ú}{{\'U}}1 {Ç}{{\c{C}}}1
}

% --- 9. LINKS E REFERÊNCIAS CRUZADAS ---
\usepackage{hyperref}
\hypersetup{
    colorlinks=true, 
    allcolors=blue,
    breaklinks=true
}
\usepackage[brazilian]{cleveref}

% --- CONFIGURAÇÕES DE NUMERAÇÃO ---
\numberwithin{equation}{chapter} 
\numberwithin{figure}{chapter}   
\numberwithin{table}{chapter}    

% 1. Formato na linha da equação: (Eq 4.1)
% Mantemos o "Eq" aqui para ele aparecer dentro dos parênteses na direita.
\renewcommand{\theequation}{Eq \thechapter.\arabic{equation}}

% 2. CONFIGURAÇÕES DO CLEVEREF
% Deixamos o nome vazio {} porque o "Eq" já vem "embutido" no número acima.
\crefname{equation}{}{}
\Crefname{equation}{}{}

% 3. Formatação da citação no texto
% Adicionamos o "Eq." com ponto aqui. 
% Como o #1 já traz "Eq 4.1", o resultado será "Eq. Eq 4.1". 
% PARA EVITAR ISSO, vamos limpar o #1 apenas na citação:
\creflabelformat{equation}{#2Eq. \thechapter.\arabic{equation}#3}

\endinput