\chapter{Sistema Internacional de Unidades}
\label{apend:unidades_SI}

\begin{table}[H]
\centering
\sffamily
\begin{threeparttable}
\caption{Grandezas e Unidades no SI para o Ensino Médio}
\label{tab:unidades_si_completa}
\begin{tabular}{llc}
\toprule
\textbf{Grandeza} & \textbf{Unidade (SI)} & \textbf{Símbolo} \\
\midrule
Comprimento & metro & \unit{m} \\
Massa & quilograma & \unit{kg} \\
Tempo & segundo & \unit{s} \\
Corrente Elétrica & ampere & \unit{A} \\
Temperatura Termodinâmica & kelvin & \unit{K} \\
Quantidade de Substância & mol & \unit{mol} \\
Intensidade Luminosa & candela & \unit{cd} \\
Área & metro quadrado & \unit{m^2} \\
Volume & metro cúbico & \unit{m^3} \\
Frequência & hertz & \unit{Hz} \\
Velocidade & metro por segundo & \unit{m/s} \\
Aceleração & metro por segundo ao quadrado & \unit{m/s^2} \\
Força & newton & \unit{N} \\
Pressão / Tensão & pascal & \unit{Pa} \\
Energia / Trabalho / Calor & joule & \unit{J} \\
Potência & watt & \unit{W} \\
Carga Elétrica & coulomb & \unit{C} \\
Diferença de Potencial (ddp) & volt & \unit{V} \\
Resistência Elétrica & ohm & \unit{\Omega} \\
Capacitância & farad & \unit{F} \\
Fluxo Magnético & weber & \unit{Wb} \\
Indução Magnética (Campo B) & tesla & \unit{T} \\
\bottomrule
\end{tabular}
\begin{tablenotes}
    \small
    \item \textbf{Nota:} Unidades como o grau Celsius (\unit{\degree C}) são aceitas pelo SI, mas a unidade base para temperatura é o Kelvin.
\end{tablenotes}
\end{threeparttable}
\end{table}
