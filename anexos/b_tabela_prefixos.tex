\chapter{Prefixos}
\label{apend:prefixos_si}

Os prefixos do SI permitem escrever quantidades muito grandes ou muito pequenas de forma simplificada, utilizando potências de base 10 \parencite{halliday2012}.

\begin{table}[H]
    \centering
    \caption{Prefixos do SI, potências e representação decimal.}
    \label{tab:prefixos_si_completa}
    \small 
    \begin{tabular}{@{}llcl@{}}
        \toprule
        \textbf{Prefixo}   & \textbf{Símbolo} & \textbf{Potência} & \textbf{Representação Decimal (Zeros)}    \\ \midrule
        Quetta             & Q                & $10^{30}$         & 1 seguido de 30 zeros                     \\
        Ronna              & R                & $10^{27}$         & 1 seguido de 27 zeros                     \\
        Yotta              & Y                & $10^{24}$         & 1.000.000.000.000.000.000.000.000         \\
        Zetta              & Z                & $10^{21}$         & 1.000.000.000.000.000.000.000             \\
        Exa                & E                & $10^{18}$         & 1.000.000.000.000.000.000                 \\
        Peta               & P                & $10^{15}$         & 1.000.000.000.000.000                     \\
        Tera               & T                & $10^{12}$         & 1.000.000.000.000 (12 zeros)              \\
        Giga               & G                & $10^{9}$          & 1.000.000.000 (9 zeros)                   \\
        Mega               & M                & $10^{6}$          & 1.000.000 (6 zeros)                       \\
        Quilo              & k                & $10^{3}$          & 1.000 (3 zeros)                           \\
        Hecto              & h                & $10^{2}$          & 100 (2 zeros)                             \\
        Deca               & da               & $10^{1}$          & 10 (1 zero)                               \\
        \textit{(unidade)} & ---              & $10^{0}$          & 1 (nenhum)                                \\
        Deci               & d                & $10^{-1}$         & 0,1 (1ª casa decimal)                     \\
        Centi              & c                & $10^{-2}$         & 0,01 (2ª casa decimal)                    \\
        Mili               & m                & $10^{-3}$         & 0,001 (3ª casa decimal)                   \\
        Micro              & $\mu$            & $10^{-6}$         & 0,000.001 (6ª casa decimal)               \\
        Nano               & n                & $10^{-9}$         & 0,000.000.001 (9ª casa decimal)           \\ 
        Pico               & p                & $10^{-12}$        & 0,000.000.000.001 (12ª casa decimal)      \\
        Femto              & f                & $10^{-15}$        & 15ª casa decimal                          \\
        Atto               & a                & $10^{-18}$        & 18ª casa decimal                          \\
        Zepto              & z                & $10^{-21}$        & 21ª casa decimal                          \\
        Yocto              & y                & $10^{-24}$        & 24ª casa decimal                          \\
        Ronto              & r                & $10^{-27}$        & 27ª casa decimal                          \\
        Quecto             & q                & $10^{-30}$        & 30ª casa decimal                          \\ \bottomrule
    \end{tabular}
\end{table}